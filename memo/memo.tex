%%%%%%%%%%%%%%%%%%%%%%%%%%%%%%%%%%%%%%%%%%%%%%%%%%%%%%%%%%%%%%%%%%%%%%
%%        Copyright (c) 2021 Carsten Wulff Software, Norway
%% %%%%%%%%%%%%%%%%%%%%%%%%%%%%%%%%%%%%%%%%%%%%%%%%%%%%%%%%%%%%%%%%%%%
%% Created       : wulff at 2021-6-13
%% %%%%%%%%%%%%%%%%%%%%%%%%%%%%%%%%%%%%%%%%%%%%%%%%%%%%%%%%%%%%%%%%%%%
%%  The MIT License (MIT)
%%
%%  Permission is hereby granted, free of charge, to any person obtaining a copy
%%  of this software and associated documentation files (the "Software"), to deal
%%  in the Software without restriction, including without limitation the rights
%%  to use, copy, modify, merge, publish, distribute, sublicense, and/or sell
%%  copies of the Software, and to permit persons to whom the Software is
%%  furnished to do so, subject to the following conditions:
%%
%%  The above copyright notice and this permission notice shall be included in all
%%  copies or substantial portions of the Software.
%%
%%  THE SOFTWARE IS PROVIDED "AS IS", WITHOUT WARRANTY OF ANY KIND, EXPRESS OR
%%  IMPLIED, INCLUDING BUT NOT LIMITED TO THE WARRANTIES OF MERCHANTABILITY,
%%  FITNESS FOR A PARTICULAR PURPOSE AND NONINFRINGEMENT. IN NO EVENT SHALL THE
%%  AUTHORS OR COPYRIGHT HOLDERS BE LIABLE FOR ANY CLAIM, DAMAGES OR OTHER
%%  LIABILITY, WHETHER IN AN ACTION OF CONTRACT, TORT OR OTHERWISE, ARISING FROM,
%%  OUT OF OR IN CONNECTION WITH THE SOFTWARE OR THE USE OR OTHER DEALINGS IN THE
%%  SOFTWARE.
%%
%%%%%%%%%%%%%%%%%%%%%%%%%%%%%%%%%%%%%%%%%%%%%%%%%%%%%%%%%%%%%%%%%%%%%%


\documentclass[technote,10pt,a4paper]{IEEEtran}

\usepackage[final]{graphicx}
\usepackage{wrapfig}
\usepackage{upgreek}
\usepackage{amssymb,amsmath}
\usepackage{cite}
\usepackage{verbatim}
\usepackage{listings}
\usepackage{xcolor}
\usepackage{flafter}
\usepackage{booktabs}
\usepackage{url}
\usepackage{textcomp}
\usepackage{dirtytalk}
\definecolor{armygreen}{rgb}{0, 0.5, 0}
\newcommand{\missing}[1]{{ #1}}
\newcommand{\edit}[1]{{ #1}}
\newcommand{\secedit}[1]{{ #1}}
\newcommand{\deleted}[1]{{ #1}}
\newcommand{\moved}[1]{{ #1}}
\newcommand{\eqn}[1]{
  \begin{equation}
    #1
  \end{equation}}

\newcommand{\qt}[1]{
  \begin{quote}
    \small
    \textit{
      #1
     }
  \end{quote}}
% -----------------------------------------------------------------
% Header
% -----------------------------------------------------------------
\begin{document}
\bibliographystyle{IEEEtran}
\title{Writing a Short Memo}
\author{Carsten~Wulff, \textit{2021-06-13}, v0.1.0 }
\maketitle

% -----------------------------------------------------------------
% Abstract
% -----------------------------------------------------------------
\begin{abstract}

  I explain why a memo should contain sections on Who, Why, How, What, and When.
\end{abstract}

% -----------------------------------------------------------------
\section{Who}
% -----------------------------------------------------------------
The reader is the \textit{Who}, and I think it's a good idea to explain what you
want from the reader.
\textit{''I'm writing this memo because I would like us all to write better memos''}.
Think of the reader when you write. Unless you're writing fiction, there is no
need to surprise the reader with twists and turns. Information should be
clear, concise, and brief. Leveraging lofty adverbs to
pontificate your pathosesque, supreme, very unique retorical skills just makes the
reader confused, and me sound like a pompus ass. And ``very unique'' doesn't
exist, it's either unique, or not.


% -----------------------------------------------------------------
\section{Why}
% -----------------------------------------------------------------
\textit{“He who has a why to live can bear almost any how” -- Nietzsche}.

Too often we jump to the \textit{How} and \textit{What} before we explain the
\textit{Why}.
Allow me to borrow an example from circuit design:
\textit{``I need you to design a single ended, fast, analog-to-digital converter with
12-bit resolution and rail-to-rail input swing''}. Those that work for you might do as you say without question, but too often, the result is not what you really need, because you forgot to explain the \textit{Why}.
\textit{``I need to measure a Wheatstone bridge \cite{wheat}. I'll measure
  the voltage on one side, and then the other, quite fast, so the samples occur
  at the same timeish. Then, I can take the difference in digital.  The signal
  between the two sides is small, 10 mV, but the signal on each side can change
  from rail-to-rail, so I need really
  high-resolution analog-to-digital converter''}.
The \textit{Why} allows the reader to question the \textit{How} or
\textit{What}. Solving the \textit{Why} is the important thing. Explain the
\textit{Why} (measuring Wheatstone bridge), and then the \textit{What} (single
ended, fast, high-resolution, analog-to-digital converter). I could then say:
\textit{``The proposed solution is stupid. It's much easier to measure differentially
across the Wheatstone bridge with a differential slow analog-to-digital
converter. Also, since the differential signal is small, we don't need
high-resolution analog-to-digital converter, just a decent common mode range''}.
Maybe I'd replace “stupid” by a more appropriate term, like ``an
interesting idea''. \textit{Why} is important, to get the right \textit{How} and \textit{What}.


% -----------------------------------------------------------------
\section{How}
% -----------------------------------------------------------------
The \textit{How} can be process, money, resources, tools, everything you need to do the \textit{What}. In the circuit example the \textit{How} could be:
\textit{``I need a test chip in Q1 next year for the first iteration of the
  analog-to-digital converter, then, about x months later, we can do the product tapeout''}.

% -----------------------------------------------------------------
\section{What}
% -----------------------------------------------------------------
A proposed solution to fix the \textit{Why}. The \textit{What} can give options,
or indeed be a single solution. It is important, however, that the \textit{What}
flows naturally from the \textit{Why}.
Expect the \textit{What} to be challenged. That's the point, we
want to find the best \textit{What}.

% -----------------------------------------------------------------
\section{When}
% -----------------------------------------------------------------
Either, ``if we start now, when can we complete the \textit{What}''. Or, ``we
need the \textit{What} to be complete by 2021-10-11, what must we do today to
get there''? Often, it’s hard to see when the \textit{What}  (become a multi
planetary spieces) to a \textit{Why} (life could end on earth) can happen,
however, one can setup an early milestone (develop vertical landing with a
rocket) that must be completed on the way. Once the first milestone is complete,
then proceed to the next, and continue until complete.

% -----------------------------------------------------------------
\section{Writing}
% -----------------------------------------------------------------
A memo is a serious text, it has a purpose. Maybe to reach a decision, provide common background information, or spark a discussion. I believe that we should strive to make memos well written. I do not claim to be a reference on writing well, for that I refer to William Zinsser’s book ``On Writing Well'' \cite{oww}.

% -----------------------------------------------------------------
\section{Conclusion}
% -----------------------------------------------------------------
A memo should contain Who, Why, How, What, and When.




\begin{thebibliography}{1}
  \providecommand{\url}[1]{#1}

  \bibitem{wheat}
  Wikipedia, ''{Wheatstone Bridge}'', online:
  \url{https://en.wikipedia.org/wiki/Wheatstone_bridge}

  \bibitem{oww}
  William Zinsser, ''{On Writing Well}'', online: \url{https://www.amazon.com/Writing-Well-Classic-Guide-Nonfiction/dp/0060891548}


\end{thebibliography}






\end{document}
